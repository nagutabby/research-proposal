\documentclass[twocolumn,article]{jlreq}
\usepackage[top=15mm, left=15mm, right=15mm, bottom=15mm]{geometry}

\usepackage{siunitx}
\usepackage{amsmath,amssymb}
\usepackage{bm}
\usepackage{mathtools}
\mathtoolsset{showonlyrefs=true}

\usepackage[hang,small,bf]{caption}
\usepackage[subrefformat=parens]{subcaption}
\captionsetup{compatibility=false}
\captionsetup[subfigure]{labelformat=simple}
\renewcommand*{\thesubfigure}{(\alph{subfigure})}

\usepackage[]{hyperref}
\hypersetup{
  bookmarksnumbered=true,
  colorlinks=false,
  pdftitle={タイトル},
  pdfauthor={作成者},
  pdfsubject={サブタイトル},
  pdfkeywords={キーワード1;キーワード2;キーワード3;}
} 

\newcommand{\MyChapter}[1]{\chapter{#1}}
\newcommand{\MySection}[1]{\section{#1}}
\newcommand{\MySubSection}[1]{\subsection{#1}}
\newcommand{\MySubSubSection}[1]{\subsubsection{#1}}

\newcommand{\MyChapRef}[1]{第~\ref{#1}~章}
\newcommand{\MySecRef}[1]{第~\ref{#1}~節}
\newcommand{\MySubSecRef}[1]{第~\ref{#1}~項}
\newcommand{\MySubSubSecRef}[1]{第~\ref{#1}~項}
\newcommand{\MyFigRef}[1]{図~\ref{#1}}
\newcommand{\MyTabRef}[1]{表~\ref{#1}}
\newcommand{\MyEqRef}[1]{式~\eqref{#1}}

\usepackage{tikz}
\usetikzlibrary{calc,intersections,positioning,shapes,arrows}
\usepackage{circuitikz}

\usepackage{tcolorbox}

\usepackage{biblatex}
\addbibresource{references.bib}

\usepackage[utf8]{inputenc}

\usepackage{newunicodechar}
\newunicodechar{、}{,}
\newunicodechar{。}{.}

\title{暗号化されたDNSトラフィックのデータによる\\悪意のあるWebサイトの分類}
\author{笹川~尋翔}

\date{\today}

\begin{abstract}
ユーザのプライバシーの保護やセキュリティの強化のために、DNSトラフィックの暗号化技術が普及している。
一方で、DNSトラフィックの暗号化は、それが悪意のあるWebサイトの通信に使用されているかどうかを判別することを困難にしている。
この研究では、暗号化されたパケットに含まれる平文のデータなどを用いて、悪意のあるWebサイトの種類を分類するモデルを構築し、評価する。
最後に、この仕組みが解決する課題と、今後解決するべき問題点を述べる。
\end{abstract}


\begin{document}

\maketitle

\section{はじめに}\label{sec:intro}
最初に普及した、UDPの53番ポートで行われる名前解決では、平文のデータが通信に用いられる。
攻撃者は、平文のDNSパケットのデータを閲覧することで、リクエストとレスポンスの平文のデータを取得し、悪用することができる。
これを踏まえ、DNSトラフィックを暗号化する技術であるDNS over TLS(DoT)、 DNS over HTTPS(DoH)、DNS over QUIC(DoQ)の活用が進められている。

しかし、既存のIPSなどのシステムでは、暗号化されたDNSパケットを用いて不正を検知することが困難である。
これは、暗号化されたDNSパケットにおいて、リクエストとレスポンスのデータが暗号化されており、検知のために利用できるデータが少ないためである。

\section{関連研究}\label{sec:relation}
2021年に行われたDNSの暗号化に関する調査\theendnotes[1]{}では、暗号化されたDNSトラフィックから得られるデータを用いて、接続先のWebサイトが悪意のあるWebサイトであるかどうかの2値に分類する手法が存在することが分かっている。
しかし、暗号化されたDNSトラフィックを利用してアクセスしたWebサイトが、悪意のあるものであった際に、そのWebサイトの種類を予測する手法はまだ存在しない。
悪意のあるWebサイトの種類とは、例えば、偽のWebサイト(フィッシングサイト)や、マルウェアを含むWebサイトなどである。

\section{研究内容}
DoTとDoHをサポートしているパブリックDNSを利用するための設定をする。
Postfixでメールサーバを構築し、フィッシングサイトに誘導するためのスパムメールを収集する。
また、Cowrieを利用して、マルウェアを配布しているドメインのデータを収集する。
暗号化されたDNSパケットに含まれる、送信元IPアドレス、宛先IPアドレス、問い合わせたレコードの種類などの平文のデータをWiresharkで収集する。
DNSトラフィックのメトリクスを収集するツールとして、DNStrace\theendnotes[2]{}ツールを使用する。
DNSの暗号化を有効にしたクライアントで、悪意のあるWebサイトに対してリクエストを送ることでレイテンシやDNSトラフィックなどのメトリクスを収集する。

次に、統計処理したデータに対して機械学習による分類を行う。
この分類では、DNSトラフィックから得られるデータを説明変数とし、悪意のあるWebサイトの種類を目的変数とする。

最後に、構築したモデルから算出される評価指標によって、モデルの性能を評価する。

\subsection{DNSの暗号化技術の選択}
現在実用化されているDNSの暗号化技術は複数存在する。
DoTは、通信を確立する際にTLSを利用することで、安全な通信を実現する。
DoHは、DoTと同様にTLSを利用するが、HTTPSプロトコルを利用するため、そのトラフィックが他のHTTPSトラフィックに紛れ込みやすく、より安全であるとされる。
DoQはLTS、HTTP/2、UDPを利用してDoTやDoHよりも高速な通信ができるように設計されている。

DoT、DoH、DoQのいずれも、1つ以上のパブリックDNSで利用できる技術であるが、特にDoTやDoHはGoogleSやCloudflareなどの多くのパブリックDNSプロバイダがサポートしている。
そのため、この研究ではDoTやDoHを利用してデータを収集する。

\subsection{Postfixによるフィッシングメールの収集}
IMAP4を用いてメールを受信するようにメールサーバを設定する。
Gmail\theendnotes[3]{}などの一部のソフトウェアは、平文での通信を行うIMAP4を使用できないため、それらのプロトコルを使用できるThunderbird\theendnotes[4]{}を電子メールクライアントとして使用する。

\subsection{Cowrieによるマルウェア配布サイトの収集}
Cowrie\theendnotes[5]{}はハニーポットの一種であり、SSH、Telnetプロトコルを利用した攻撃を収集して分析することができる。
これを使用して、攻撃者がマルウェアをダウンロードした際のURLを記録し、それを元にそれぞれのドメインのDNSトラフィックのデータを収集する。

\subsection{DNStraceによるDNSトラフィックのメトリクスの計測}
DNStraceでは、ネームサーバへ名前解決要求を繰り返し行い、レイテンシの分布を計測することができる。
A、AAAA、TXTレコードに対して、名前解決要求を100回行い、それぞれのレイテンシの分布を統計量として求める。


\subsection{分類手法の選択}
最適な分類手法は、データの分布により異なる。
ロジスティック回帰分析、k近傍法、サポートベクターマシン、決定木は、データの散らばりが大きい場合でも比較的高い精度で分類できる。
この研究では、モデルの説明のしやすさと実装のしやすさを重視し、決定木のアンサンブル学習の手法であるランダムフォレストと勾配ブースティング決定木を利用する。

\section{評価方法}
分類の性能を評価するために用いる指標を決める際には、偽陽性率と偽陰性率の中からどの比率を重視するかを考慮する。
この研究では、偽陽性率と偽陰性率の両方を重視するため、評価指標の1つとしてF値を用いる。
その他の指標として、予測の確実性を算出するためにLogLossを用い、分類の正確さを算出するために正解率を用いる。
また、データの分布をヒストグラムによって視覚化して比較することで、それぞれのデータの傾向を考察する。

\section{研究結果}
ボットネットでは、オリジンサーバを隠蔽するためにTTLの値が小さく設定されていたり、ファストフラックスと呼ばれるIPアドレスの関連付けを高速に切り替える手法が使用されていたりする場合がある。
そのため、マルウェアを含むWebサイトと分類される場合の特徴的なデータは、レイテンシとゾーン情報であると考えられる。

偽のWebサイトは、マルウェアを含むWebサイトと比べてWebサイトのコンテンツが非常に多く、その分、1アクセスあたりの負荷が高いことが考えられる。
また、DDNSを利用したフィッシングサイトが多いことから、特定のネームサーバはフィッシングサイトに使われやすい可能性がある。
そのため、偽のWebサイトと分類される場合の特徴的なデータは、レイテンシやネームサーバの種類であると考えられる。


\section{まとめ}
暗号化されたDNSトラフィックを利用して悪意のあるWebサイトの種類を分類することで、DNSトラフィックを暗号化した高度な攻撃を分析しやすくなると考えられる。

\section*{参考文献}
% \printbibliography

\theendnotes[1]{MINZHAO LYU, HASSAN HABIBI GHARAKHEILI, and VIJAY SIVARAMAN, 2021. A Survey on DNS Encryption: Current Development, Malware Misuse, and Inference Techniques, https://dl.acm.org/doi/epdf/10.1145/3547331}

\theendnotes[2]{https://github.com/redsift/dnstrace}

\theendnotes[3]{https://www.google.com/intl/ja/gmail/about/}

\theendnotes[4]{https://www.thunderbird.net/ja/}

\theendnotes[5]{https://github.com/cowrie/cowrie}

\end{document}

