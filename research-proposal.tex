\documentclass[twocolumn,article]{jlreq}
\usepackage{sbase-prop}
\usepackage[top=15mm, left=15mm, right=15mm, bottom=15mm, includefoot]{geometry}

%------------------------------
\title{ここにタイトルを書く長いときは\\このようにして2行目に書く}%
\author{名字~氏名}%
%\author{名字~氏名\\所属}%
%\author{名字~氏名\thanks{E-mail: HogeHoge@hoge.ac.jp}\\所属}%
%\date{}%							空白にすることで省略可能

%\date{\today}%


\begin{abstract}
おおむね5行程度で背景と目的,手段,それによって何がどうなるのかを簡潔に記述する.
おおむね5行程度で背景と目的,手段,それによって何がどうなるのかを簡潔に記述する.
おおむね5行程度で背景と目的,手段,それによって何がどうなるのかを簡潔に記述する.
おおむね5行程度で背景と目的,手段,それによって何がどうなるのかを簡潔に記述する.
おおむね5行程度で背景と目的,手段,それによって何がどうなるのかを簡潔に記述する.
おおむね5行程度で背景と目的,手段,それによって何がどうなるのかを簡潔に記述する.
\end{abstract}

%------------------------------

\begin{document}

% ここから本文

\maketitle%					%abstactより後にmaketitleを置くと,アブストラクトだけ段組にならない

\section{はじめに}\label{sec:intro}
段落を変えるときは以下のようにする.

はじめに背景や研究の目的,見通しや手段をできる限り簡潔に記述する.

技術的な課題もこの節で述べておく.

最終提出のリサーチプロポーザルではこのくらい記述する.
最終提出のリサーチプロポーザルではこのくらい記述する.
最終提出のリサーチプロポーザルではこのくらい記述する.
最終提出のリサーチプロポーザルではこのくらい記述する.
最終提出のリサーチプロポーザルではこのくらい記述する.
最終提出のリサーチプロポーザルではこのくらい記述する.
最終提出のリサーチプロポーザルではこのくらい記述する.
最終提出のリサーチプロポーザルではこのくらい記述する.
最終提出のリサーチプロポーザルではこのくらい記述する.
最終提出のリサーチプロポーザルではこのくらい記述する.
最終提出のリサーチプロポーザルではこのくらい記述する.
最終提出のリサーチプロポーザルではこのくらい記述する.
最終提出のリサーチプロポーザルではこのくらい記述する.
最終提出のリサーチプロポーザルではこのくらい記述する.
最終提出のリサーチプロポーザルではこのくらい記述する.
最終提出のリサーチプロポーザルではこのくらい記述する.
最終提出のリサーチプロポーザルではこのくらい記述する.


\section{関連研究}\label{sec:relation}
本稿との関連する従来の研究や,
他のグループなどで発表されている内容と,
自分の研究内容を関連させ,
本稿では何を狙っているのか,関連研究との差異はどこにあるのかを説明する.
関連する研究を掲載する場合は参考文献に記述し参照しておく

研究内容に関連する技術について解説することがあればここで書いておく.

\LaTeX 文書中で表や図の参照を入れたいときはとのようにする.
複数の図を$1$つにまとめたいときはのように,とのようにする.
また,数式に関してはのように書く.
\begin{align}
  \label{eq:normal-distribution}
  \begin{aligned}
    f(x)=\frac{1}{\sqrt{2\pi\sigma^{2}}}\exp\left( - \frac{\left( x - \mu \right)^2}{2\sigma^2} \right)
  \end{aligned}
\end{align}


最終提出のリサーチプロポーザルではこのくらい記述する.
最終提出のリサーチプロポーザルではこのくらい記述する.
最終提出のリサーチプロポーザルではこのくらい記述する.
最終提出のリサーチプロポーザルではこのくらい記述する.
最終提出のリサーチプロポーザルではこのくらい記述する.
最終提出のリサーチプロポーザルではこのくらい記述する.
最終提出のリサーチプロポーザルではこのくらい記述する.
最終提出のリサーチプロポーザルではこのくらい記述する.
最終提出のリサーチプロポーザルではこのくらい記述する.


\section{研究内容}%%change later
研究内容について書く.

\subsection{小節の書き方}
\LaTeX ではsubsectionを用いて小節を記述する.

\subsubsection{小々節の書き方}
\LaTeX ではsubsubsectionを用いて小々節を記述する.

\subsection{注意点}
要旨の作成において,次の$4$つに注意し作成せよ.
\begin{enumerate}
  \item 各セクションの名前は適宜変更する
  \item すべてのファイルは卒論をまとめるときに用いるので,各自管理しておく
  \item 画像を作成する場合は,画像内の文章は英語フォントのTimes系や一般的なフォントを用いる
  \item 「、や。」ではなく全角の「,や.」を用いる
  \item 体言止めは使用しない
\end{enumerate}


最終提出のリサーチプロポーザルではこのくらい記述する.
最終提出のリサーチプロポーザルではこのくらい記述する.
最終提出のリサーチプロポーザルではこのくらい記述する.
最終提出のリサーチプロポーザルではこのくらい記述する.
最終提出のリサーチプロポーザルではこのくらい記述する.




\section{評価方法}
提案する研究はどのように評価するのか記述する.

提案内容によっては評価方法は未定でも構わない.

最終提出のリサーチプロポーザルではこのくらい記述する.
最終提出のリサーチプロポーザルではこのくらい記述する.
最終提出のリサーチプロポーザルではこのくらい記述する.
最終提出のリサーチプロポーザルではこのくらい記述する.
最終提出のリサーチプロポーザルではこのくらい記述する.
最終提出のリサーチプロポーザルではこのくらい記述する.
最終提出のリサーチプロポーザルではこのくらい記述する.
最終提出のリサーチプロポーザルではこのくらい記述する.
最終提出のリサーチプロポーザルではこのくらい記述する.


\section{研究結果}
どのような結果が予想できるのか記述する.
研究内容によっては結果の予測はできない場合もある.

最終提出のリサーチプロポーザルではこのくらい記述する.
最終提出のリサーチプロポーザルではこのくらい記述する.
最終提出のリサーチプロポーザルではこのくらい記述する.

\section{まとめ}
まとめとして,本稿の内容を凝縮し,簡潔に述べる.

また,今後の課題などを述べる.

最終提出のリサーチプロポーザルではこのくらい記述する.
最終提出のリサーチプロポーザルではこのくらい記述する.
最終提出のリサーチプロポーザルではこのくらい記述する.



%%%%%%%%%%%%%%%%%%%%%%%%%%%%%%%%%%%%
%%%%
%%%%		表の挿入
%%%%
%%%%%%%%%%%%%%%%%%%%%%%%%%%%%%%%%%%
%	{\tabcolsep = 2mm%%		表の横幅余白
%	  {\renewcommand\arraystretch{1.2}
\begin{table}[tbp]\centering\caption{表のタイトル}\label{tab:parameters}
  \begin{tabular}{c|c}\hline					%{c|c} で表の要素の配置を決める.
    目的                 & 方法                    \\ \hline %\hline 
    数式を書きたい       & $ここ$ のように\$で囲む \\
    表を作りたい         & table環境を使う         \\
    画像を挿入したい     & figure環境を使う        \\
    注脚をつけたい       & footnoteを使う          \\
    細かい設定を変えたい & 自分で調べる~           \\
    \hline
  \end{tabular}
\end{table}
%\end{spacing}
%}			%\renewcommand\arraystretch{}の綴じ
%}			%\tabcolsepの綴じ

%%%%%%%%%%%%%%%%%%%%%%%%%%%%%%%%%%%%
%%%%
%%%%		 図の挿入
%%%%
%%%%%%%%%%%%%%%%%%%%%%%%%%%%%%%%%%%%	
\begin{figure}[tbp]\centering
\end{figure}
%



%%%%%%%%%%%%%%%%%%%%%%%%%%%%%%%%%%%%
%%%%
%%%%		 複数の図の挿入(subfigure)
%%%%
%%%%%%%%%%%%%%%%%%%%%%%%%%%%%%%%%%%%
\begin{figure}[tbp] \centering
  \begin{subfigure}[b]{0.45\columnwidth}\centering
  \end{subfigure}
  \begin{subfigure}[b]{0.45\columnwidth}\centering
  \end{subfigure}
  \caption{\texttt{subfigure}の使い方}
  \label{fig:demo}
\end{figure}

%%%%%%%%%%%%%%%%%%%%%%%%%%%%%%%%%%%%
%%%%
%%%%		 外部ファイルからの参考文献の挿入(bibtex)
%%%%
%%%%%%%%%%%%%%%%%%%%%%%%%%%%%%%%%%%%	
\bibliographystyle{junsrt}		%出てきた順索引 
%\bibliographystyle{jplain}		%アルファベット順索引
\bibliography{Ref}
%
%%%%%%%%%%%%%%%%%%%%%%%%%%%%%%%%%%%%
%%%%
%%%%		 直接入力の参考文献の挿入(bibtex)
%%%%
%%%%%%%%%%%%%%%%%%%%%%%%%%%%%%%%%%%%	
%---------------------------------
%\begin{thebibliography}{9}
%\bibitem{skmtlab} SakamotoLab., \url{http://sskmtlab.mydns.jp/}, (最終アクセス日:2021年12月24日).
%\bibitem{google} Google Inc., \url{http://www.google.com/}, (最終アクセス日:2021年12月24日).
%\end{thebibliography}
%---------------------------------
%\begin{flushright}
%{\bfseries 指導教官[坂本~真仁]}
%\end{flushright}


\end{document}

